O TM4C1294NCPDT possui 15 portas GPIOs de 8 pinos cada. Elas são nomeadas com as letras de \emph{'A'} à \emph{'Q'} menos as letras \emph{'I'} e \emph{'O'}. Algumas das especificações das GPIOs são:

\begin{itemize}
	\item Possui mais de 90 GPIOS, dependendo da configuração usada.
	\item Pinos específicos possuem ligação com os periféricos do microcontrolador e suas funções devem ser configuradas.
	\item Tensão em configuração de entrada de 3,3 V.
	\item Todas as portas são conectadas ao Barramento de Alta Performance (AHB).
	\item Mudança rápida de nível de saída da porta a cada ciclo de clock em portas ligadas ao AHB.
	\item Interrupções por pinos nas portas P e Q por bordas de subida, descida ou ambas.
	\item Podem ser usadas para iniciar uma sequência de amostragem do A/D ou uma transferência $\mu$DMA.
	\item Estado dos pinos podem ser mantidos durante o modo de hibernação; variações de nível nos pinos da porta P podem ser usadas para acordar o sistema da hibernação.
	\item Pinos configurados como entradas digital utilizam circuitos Schmitt-trigger.
	\item Pinos possuem resistores de pull-up e pull-down e limites de corrente para 2, 4, 6, 8, 10 e 12 mA.
	\item Configuração dreno-aberto habilitada
\end{itemize}

\section{Na TivaWare}

As funções presentes na TivaWare responsáveis pela configuração e manipulação das GPIOs são listadas a seguir.

\begin{lstlisting}[style=funcao]
	void SysCtlMOSCConfigSet(uint32_t ui32Config)
\end{lstlisting}

Configura o circuito monitor do oscilador principal.

\begin{description}
	\item [\ttbu{ui32Config}]\hfill \\
	Configura o controle do oscilador principal a partir da lógica OR de definições no formato \textbf{SYSCTL\_MOSC\_\emph{k}}, onde \textbf{\emph{k}} pode assumir o valor de:
	\begin{itemize}
		\item \textbf{VALIDATE} para verificar uma falha do MOSC
		\item \textbf{INTERRUPT} quando se deseja gerar uma interrupção ao invés do reset do processador
		\item \textbf{NO\_XTAL} se não há um oscilador esxterno nos pinos OSC0/OSC1, reduzindo o consumo de energia
		\item \textbf{PWR\_DIS} se deseja-se que o MOSC seja desligado. Se este parâmetro não for especificado o oscilador permanece ligado
		\item \textbf{LOWFREQ} se a frequência do MOSC está abaixo de 10 MHZ
		\item \textbf{HIGHFREQ} se a frequência do MOSC está acima de 10 MHZ
		\item \textbf{SESRC} quando o MOSC é um oscilador \emph{single-end} conectado ao pino OSC0. Se não especificado, assume-se que um cristal está em uso.
	\end{itemize}

\end{description}

\section{Exemplo}