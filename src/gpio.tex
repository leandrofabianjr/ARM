O TM4C1294NCPDT possui 15 portas GPIOs de 8 pinos cada. Elas são nomeadas com as letras de \emph{'A'} à \emph{'Q'} menos as letras \emph{'I'} e \emph{'O'}. Algumas das especificações das GPIOs são:

\begin{itemize}
	\item Possui mais de 90 GPIOS, dependendo da configuração usada.
	\item Pinos específicos possuem ligação com os periféricos do microcontrolador e suas funções devem ser configuradas.
	\item Tensão em configuração de entrada de 3,3 V.
	\item Todas as portas são conectadas ao Barramento de Alta Performance (AHB).
	\item Mudança rápida de nível de saída da porta a cada ciclo de clock em portas ligadas ao AHB.
	\item Interrupções por pinos nas portas P e Q por bordas de subida, descida ou ambas.
	\item Podem ser usadas para iniciar uma sequência de amostragem do A/D ou uma transferência $\mu$DMA.
	\item Estado dos pinos podem ser mantidos durante o modo de hibernação; variações de nível nos pinos da porta P podem ser usadas para acordar o sistema da hibernação.
	\item Pinos configurados como entradas digital utilizam circuitos Schmitt-trigger.
	\item Pinos possuem resistores de pull-up e pull-down e limites de corrente para 2, 4, 6, 8, 10 e 12 mA.
	\item Configuração dreno-aberto habilitada
\end{itemize}

\section{Na TivaWare}

As principais funções presentes na TivaWare responsáveis pela configuração e manipulação das GPIOs são listadas a seguir.

\begin{lstlisting}[style=funcao]
	void GPIODirModeSet(uint32_t ui32Port,
						uint8_t ui8Pins,
						uint32_t ui32PinIO)
\end{lstlisting}

Configura a direção dos pinos da porta especificada.

\begin{description}
	\item [\ttbu{ui32Port}]\hfill \\
	Base da porta a ser configurada. Normalmente \textbf{GPIO\_PORT\emph{k}\_BASE}, onde \textbf{\emph{k}} é a letra identificadora da porta.
	
	\item [\ttbu{ui32Pins}]\hfill \\
	Pacote em OU binário de pinos com 8 bits. Onde o bit 0 representa o pino 0, o bit 1 representa o pino 1 e assim por diante. É possível utilizar das definições no formato  \textbf{GPIO\_PIN\_\emph{k}}, onde \textbf{\emph{k}} é o número do pino de 0 a 7.
	
	\item [\ttbu{ui32PinIO}]\hfill \\
	Direção ou modo dos pinos especificados.
	
	\begin{description}
		\item [\textbf{GPIO\_DIR\_MODE\_IN}] pinos configurados como de entrada
		\item [\textbf{GPIO\_DIR\_MODE\_OUT}] pinos configurados como de saída
		\item [\textbf{GPIO\_DIR\_MODE\_HW}] direção dos pinos controlada pelo pelo hardware de algum periférico
	\end{description}
\end{description}

\begin{lstlisting}[style=funcao]
	void GPIOPinConfigure(uint32_t ui32PinConfig)
\end{lstlisting}

Configura pino utilizado por algum periférico especial.

\begin{description}
	\item [\ttbu{ui32PinConfig}]\hfill \\
	Valor de configuração do pino, especificado pelas definições no formato \\ \textbf{GPIO\_P\emph{Ak}\_\emph{pin}}. Sendo \textbf{\emph{A}} uma letra representando a porta que contém o pino, \textbf{\emph{k}} o número do pino referente à porta e \textbf{\emph{pin}} a função atribuída ao pino.
\end{description}

\begin{lstlisting}[style=funcao]
	uint32_t GPIODirModeGet(uint32_t ui32Port,
							uint8_t ui8Pin)
\end{lstlisting}

Retorna a direção de pinos de uma determinada porta.

\begin{description}
	\item [\ttbu{ui32Port}]\hfill \\
	Base da porta a ser consultada. Normalmente \textbf{GPIO\_PORT\emph{k}\_BASE}, onde \textbf{\emph{k}} é a letra identificadora da porta.
	
	\item [\ttbu{ui32Pins}]\hfill \\
	Pacote em OU binário de pinos com 8 bits. Onde o bit 0 representa o pino 0, o bit 1 representa o pino 1 e assim por diante. É possível utilizar das definições no formato  \textbf{GPIO\_PIN\_\emph{k}}, onde \textbf{\emph{k}} é o número do pino de 0 a 7.
\end{description}

\begin{lstlisting}[style=funcao]
	uint32_t GPIOPadConfigSet(uint32_t ui32Port,
							  uint8_t ui8Pins,
							  uint32_t ui32Strength,
							  uint32_t ui32PinType)
\end{lstlisting}

Configura especificações de potência e tipo dos pinos das portas GPIOs.

\begin{description}
	\item [\ttbu{ui32Port}]\hfill \\
	Base da porta a ser configurada. Normalmente \textbf{GPIO\_PORT\emph{k}\_BASE}, onde \textbf{\emph{k}} é a letra identificadora da porta.
	
	\item [\ttbu{ui32Pins}]\hfill \\
	Pacote em OU binário de pinos com 8 bits. Onde o bit 0 representa o pino 0, o bit 1 representa o pino 1 e assim por diante. É possível utilizar das definições no formato  \textbf{GPIO\_PIN\_\emph{k}}, onde \textbf{\emph{k}} é o número do pino de 0 a 7.
	
	\item [\ttbu{ui32Strength}]\hfill \\
	Especifica a corrente máxima que os pinos podem fornecer. Configurada através da definição \textbf{GPIO\_STRENGTH\_\emph{k}MA} onde o máximo de corrente nos pinos é de \textbf{\emph{k}} mA. Sendo que \textbf{\emph{k}} pode assumir os valores: 2, 4, 6, 8, 10 e 12.
	
	\item [\ttbu{ui32PinType}]\hfill \\
	Valor que configura o tipo do pino. Definido por \textbf{GPIO\_PIN\_TYPE\_\emph{k}}, onde k pode assumir os seguintes valores:
	\begin{description}
		\item [\textbf{STD}] pino do tipo \emph{push-pull} (tipo padrão)
		\item [\textbf{STD\_WPU}] pino do tipo \emph{pull-up} fraco, ou seja, com um resistor de \emph{pull-up} pouca corrente circulando
		\item [\textbf{STD\_WPD}] pino do tipo \emph{pull-down} fraco, ou seja, com um resistor de \emph{pull-down} pouca corrente circulando
		\item [\textbf{OD}] pino do tipo coletor aberto
		\item [\textbf{ANALOG}] pino de entrada analógica
	\end{description}
\end{description}

\begin{lstlisting}[style=funcao]
	uint32_t GPIOPadConfigGet(uint32_t ui32Port,
							  uint8_t ui8Pin,
							  uint32_t *pui32Strength,
							  uint32_t *pui32PinType)
\end{lstlisting}

Retorna as informações de configuração utilizadas no pino de uma determinada porta.

\begin{description}
	\item [\ttbu{ui32Port}]\hfill \\
	Base da porta a ser consultada. Normalmente \textbf{GPIO\_PORT\emph{k}\_BASE}, onde \textbf{\emph{k}} é a letra identificadora da porta.
	
	\item [\ttbu{ui32Pins}]\hfill \\
	Pacote em OU binário de pinos com 8 bits. Onde o bit 0 representa o pino 0, o bit 1 representa o pino 1 e assim por diante. É possível utilizar das definições no formato  \textbf{GPIO\_PIN\_\emph{k}}, onde \textbf{\emph{k}} é o número do pino de 0 a 7.
	
	\item [\ttbu{pui32Strength}]\hfill \\
	Ponteiro de um endereço de memória já alocado, onde estarão contidas as especificações de corrente.
	
	\item [\ttbu{pui32PinType}]\hfill \\
	Ponteiro de um endereço de memória já alocado, onde estarão contidas as especificações do tipo do pino.
\end{description}

\begin{lstlisting}[style=funcao]
	int32_t GPIOPinRead(uint32_t ui32Port,
						uint8_t ui8Pins)
\end{lstlisting}

Retorna os valores contidos nos pinos de entrada especificados.

\begin{description}
	\item [\ttbu{ui32Port}]\hfill \\
	Base da porta a ser consultada. Normalmente \textbf{GPIO\_PORT\emph{k}\_BASE}, onde \textbf{\emph{k}} é a letra identificadora da porta.
	
	\item [\ttbu{ui32Pins}]\hfill \\
	Pacote em OU binário de pinos com 8 bits. Onde o bit 0 representa o pino 0, o bit 1 representa o pino 1 e assim por diante. É possível utilizar das definições no formato  \textbf{GPIO\_PIN\_\emph{k}}, onde \textbf{\emph{k}} é o número do pino de 0 a 7.
\end{description}

\begin{lstlisting}[style=funcao]
	void GPIOPinWrite(uint32_t ui32Port,
					  uint8_t ui8Pins,
					  uint8_t ui8Val)
\end{lstlisting}

Configura os valores nos pinos de saída especificados.

\begin{description}
	\item [\ttbu{ui32Port}]\hfill \\
	Base da porta a ser configurada. Normalmente \textbf{GPIO\_PORT\emph{k}\_BASE}, onde \textbf{\emph{k}} é a letra identificadora da porta.
	
	\item [\ttbu{ui32Pins}]\hfill \\
	Pacote em OU binário de pinos com 8 bits. Onde o bit 0 representa o pino 0, o bit 1 representa o pino 1 e assim por diante. É possível utilizar das definições no formato  \textbf{GPIO\_PIN\_\emph{k}}, onde \textbf{\emph{k}} é o número do pino de 0 a 7.
	
	\item [\ttbu{ui32Val}]\hfill \\
	Pacote de valores de cada pino, contendo 8 bits. Onde o bit 0 representa sinal alto no pino 0, o bit 1 representa sinal alto no pino 1 e assim por diante. É possível utilizar das definições no formato  \textbf{GPIO\_PIN\_\emph{k}}, onde \textbf{\emph{k}} é o número do pino de 0 a 7.
\end{description}

\begin{lstlisting}[style=funcao]
	void GPIOIntTypeSet(uint32_t ui32Port,
						uint8_t ui8Pins,
						uint32_t ui32IntType)
\end{lstlisting}

Configura os tipos das interrupções nos pinos de entrada especificados.

\begin{description}
	\item [\ttbu{ui32Port}]\hfill \\
	Base da porta a ser configurada. Normalmente \textbf{GPIO\_PORT\emph{k}\_BASE}, onde \textbf{\emph{k}} é a letra identificadora da porta.
	
	\item [\ttbu{ui32Pins}]\hfill \\
	Pacote em OU binário de pinos com 8 bits. Onde o bit 0 representa o pino 0, o bit 1 representa o pino 1 e assim por diante. É possível utilizar das definições no formato  \textbf{GPIO\_PIN\_\emph{k}}, onde \textbf{\emph{k}} é o número do pino de 0 a 7.
	
	\item [\ttbu{ui32IntType}]\hfill \\
	Definição do evento que dispara a interrupção.
	\begin{description}
		\item [\textbf{GPIO\_FALLING\_EDGE}] disparo de interrupção na borda de descida
		\item [\textbf{GPIO\_RISING\_EDGE}] disparo de interrupção na borda de subida
		\item [\textbf{GPIO\_BOTH\_EDGES}] disparo de interrupção em ambas as bordas
		\item [\textbf{GPIO\_LOW\_LEVEL}] disparo de interrupção quando em nível baixo
		\item [\textbf{GPIO\_HIGH\_LEVEL}] disparo de interrupção quando em nível alto
	\end{description}
\end{description}

\begin{lstlisting}[style=funcao]
	void GPIOIntEnable(uint32_t ui32Port,
					   uint32_t ui32IntFlags)
\end{lstlisting}

Habilita as interrupções nos pinos especificados.

\begin{description}
	\item [\ttbu{ui32Port}]\hfill \\
	Base da porta a ser configurada. Normalmente \textbf{GPIO\_PORT\emph{k}\_BASE}, onde \textbf{\emph{k}} é a letra identificadora da porta.
	
	\item [\ttbu{ui32IntFlags}]\hfill \\
	Pacote em OU binário de pinos com 8 bits. Onde o bit 0 representa o pino 0, o bit 1 representa o pino 1 e assim por diante. É possível utilizar das definições no formato  \textbf{GPIO\_INT\_PIN\_\emph{k}}, onde \textbf{\emph{k}} é o número do pino de 0 a 7.
\end{description}

\begin{lstlisting}[style=funcao]
	void GPIOIntDisable(uint32_t ui32Port,
					   uint32_t ui32IntFlags)
\end{lstlisting}

Desabilita as interrupções nos pinos especificados.

\begin{description}
	\item [\ttbu{ui32Port}]\hfill \\
	Base da porta a ser configurada. Normalmente \textbf{GPIO\_PORT\emph{k}\_BASE}, onde \textbf{\emph{k}} é a letra identificadora da porta.
	
	\item [\ttbu{ui32IntFlags}]\hfill \\
	Pacote em OU binário de pinos com 8 bits. Onde o bit 0 representa o pino 0, o bit 1 representa o pino 1 e assim por diante. É possível utilizar das definições no formato  \textbf{GPIO\_INT\_PIN\_\emph{k}}, onde \textbf{\emph{k}} é o número do pino de 0 a 7.
\end{description}

\begin{lstlisting}[style=funcao]
	void GPIOIntClear(uint32_t ui32Port,
					   uint32_t ui32IntFlags)
\end{lstlisting}

Limpa o \emph{buffer} que armazena se já houve uma interrupção nos pinos especificados.

\begin{description}
	\item [\ttbu{ui32Port}]\hfill \\
	Base da porta a ser configurada. Normalmente \textbf{GPIO\_PORT\emph{k}\_BASE}, onde \textbf{\emph{k}} é a letra identificadora da porta.
	
	\item [\ttbu{ui32IntFlags}]\hfill \\
	Pacote em OU binário de pinos com 8 bits. Onde o bit 0 representa o pino 0, o bit 1 representa o pino 1 e assim por diante. É possível utilizar das definições no formato  \textbf{GPIO\_INT\_PIN\_\emph{k}}, onde \textbf{\emph{k}} é o número do pino de 0 a 7.
\end{description}

\begin{lstlisting}[style=funcao]
	void GPIOIntRegister(uint32_t ui32Port,
						 void (*pfnIntHandler)(void))
\end{lstlisting}

Configura a função de tratamento de uma interrupção da GPIO. Aquela que é chamada quando ocorre a interrupção.

\begin{description}
	\item [\ttbu{ui32Port}]\hfill \\
	Base da porta a ser configurada. Normalmente \textbf{GPIO\_PORT\emph{k}\_BASE}, onde \textbf{\emph{k}} é a letra identificadora da porta.
	
	\item [\ttbu{pfnIntHandler}]\hfill \\
	Ponteiro da função de tratamento. Esta não deve receber nada como parâmetro e nem retornar nada.
\end{description}

\section{Exemplo}

Um exemplo básico de configuração das portas GPIOs do microcontrolador é dado a seguir:

\begin{lstlisting}[style=citacao]

// Configura os pinos 0, 2 e 7 da porta A como sendo de entrada
GPIODirModeSet(
	GPIO_PORTA_BASE, GPIO_PIN_0 | GPIO_PIN_2 |
	GPIO_PIN_7, GPIO_DIR_MODE_IN);

// Configura os pinos 4, 5 e 6 da porta A como sendo de saida
GPIODirModeSet(GPIO_PORTA_BASE, GPIO_PIN_4 | 
	GPIO_PIN_5 | GPIO_PIN_6, GPIO_DIR_MODE_OUT);

// Configura os pinos 4 e 6 da porta A para poder 
// fornecer ate 4 mA para o circuito
GPIOPadConfigSet(GPIO_PORTA_BASE, 
	GPIO_PIN_4 | GPIO_PIN_6, GPIO_STRENGTH_4MA);

// Armazena valor de entrada do pino 2 da porta A
int valor = GPIOPinRead(GPIO_PORTA_BASE, GPIO_PIN_2);

// Configura pino de saida 4 em sinal alto e o 6 em baixo
GPIOPinWrite(
	GPIO_PORTA_BASE, GPIO_PIN_4 | GPIO_PIN_6, GPIO_PIN_4);

\end{lstlisting}


