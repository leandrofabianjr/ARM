
Temporizadores em microcontroladores são usados como contadores de intervalo de tempo para eventos internos e externos ao DSP. Tais temporizadores possibilitam provocar interrupções a tempos programáveis, sendo esse recurso essencial em muitas aplicações.  

\section{GPTM no TM4C1294NCPDT}

O temporizador de propósito geral, ou \emph{General-Purpose Timers} (GPTM),  do Tiva - TM4C129NCPDT possui blocos de contadores de 16/32 bits. Em cada bloco há dois contadores de 16 bits, sendo um do \emph{Timer A} e o outro do \emph{Timer B}, que podem ser concatenados em apenas um contador de 32 bits.

Qualquer temporizador do Tiva é capaz de ser usado como um gatilho para conversões do ADC. Porém não se pode usar mais de um temporizador como gatilho para o ADC.

O GPTM é o recurso temporizador padrão no Tiva, porém ele não é o único temporizador presente. Existem os temporizadores do módulo PWM, do módulo \emph{WatchDog Timer}, e do módulo \emph{Systick}, que também podem ser usados de modo parecido ao GPTM. 

Dentro do GPTM há 8 blocos de contadores de 16/32 bits. A fonte de clock para os contadores pode ser selecionada entre o clock do sistema ou o ALTCLK (\emph{Alternate CLock}). Tendo em vista que o ALTCLK pode ser proveniente do PIOSC, clock do módulo de hibernação ou o oscilador de baixa frequência (LFIOSC).  As funcionalidades básicas dos contadores de cada bloco são: 

\begin{itemize}
	\item Provocar uma única interrupção após intervalo de tempo programável, com contador de 16/32 bits.
	\item Provocar interrupções periódicas a intervalo de tempo programável, com contador de 16/32 bits.
	\item Clock de tempo real utilizando clock externo de 32,768 kHz, com contador de 32 bits
	\item Captura de intervalo de tempo através de detecção de borda de sinal externo com divisor de clock de 8 bits, com contador de 16 bits. 
	\item Modo PWM, com divisor de clock de 8 bits, com contador de 16 bits.
	\item Contador do modo contagem crescente ou decrescente.
	\item Modo de captura e comparação, com contador de 16/32 bits.
	\item Encadeamento de temporizadores para múltiplos disparos de interrupção.
	\item Gatilho para disparo do ADC.
\end{itemize}

A Tabela \ref{tab:CanaisTimer} apresenta os pinos de entrada e saída do GPTM para os modos de funcionamento captura, comparação e geração de PWM.

\begin{center}
	\begin{longtable}{|c|c|c|c|c|}
		\rowcolor[HTML]{000000}
		{\color[HTML]{FFFFFF} Pino} & {\color[HTML]{FFFFFF} $n^{o}$} & {\color[HTML]{FFFFFF} Mux/Função} & {\color[HTML]{FFFFFF} Tipo} & {\color[HTML]{FFFFFF} Descrição}            \\
		\multirow{3}{*}{T0CCP0}    & 1   & PD0 (3) & \multirow{3}{*}{I/O} & \multirow{3}{*}{Timer 0 Captura/Comparação/PWM 0}\\
		& 33  & PA0 (3) &     &                                 \\
		& 85  & PL4 (3) &     &                                 \\ \hline
		\multirow{3}{*}{T0CCP1}    & 2   & PA1 (6) & \multirow{3}{*}{I/O} & \multirow{3}{*}{Timer 0 Captura/Comparação/PWM 1}\\
		& 34  & PL5 (3) &     &                                 \\
		& 86  & PL5 (3) &     &                                 \\ \hline
		\multirow{3}{*}{T1CCP0}    & 3   & PD2 (3) & \multirow{3}{*}{I/O} & \multirow{3}{*}{Timer 1 Captura/Comparação/PWM 0}\\
		& 35  & PA2 (3) &     &                                 \\
		& 94  & PL6 (3) &     &                                 \\ \hline
		\multirow{3}{*}{T1CCP1}    & 4   & PD3 (3) & \multirow{3}{*}{I/O} & \multirow{3}{*}{Timer 1 Captura/Comparação/PWM 1}\\
		& 36  & PA3 (3) &     &                                 \\
		& 93  & PL7 (3) &     &                                 \\ \hline
		\multirow{2}{*}{T2CCP0}   & 37  & PA4 (3) & \multirow{2}{*}{I/O} & \multirow{2}{*}{Timer 2 Captura/Comparação/PWM 0}\\
		& 78  & PM0 (3) &     &                                 \\ \hline
		\multirow{2}{*}{T2CCP1}    & 38  & PA5 (3) & \multirow{2}{*}{I/O} & \multirow{2}{*}{Timer 2 Captura/Comparação/PWM 1}\\
		& 77  & PM1 (3) &     &                                 \\ \hline
		\multirow{3}{*}{T3CCP0}    & 40  & PA6 (3) & \multirow{3}{*}{I/O} & \multirow{3}{*}{Timer 3 Captura/Comparação/PWM 0}\\
		& 76  & PM2 (3) &     &                                 \\ 
		& 125 & PD4 (3) &     &                                 \\ \hline
		\multirow{3}{*}{T3CCP1}    & 41  & PA7 (3) & \multirow{3}{*}{I/O} & \multirow{3}{*}{Timer 3 Captura/Comparação/PWM 1}\\
		& 75  & PM3 (3) &     &                                 \\ 
		& 126 & PD5 (3) &     &                                 \\ \hline
		\multirow{3}{*}{T4CCP0}    & 74  & PM4 (3) & \multirow{3}{*}{I/O} & \multirow{3}{*}{Timer 4 Captura/Comparação/PWM 0}\\
		& 95  & PB0 (3) &     &                                 \\ 
		& 127 & PD6 (3) &     &                                 \\ \hline
		\multirow{3}{*}{T4CCP1}    & 73  & PM5 (3) & \multirow{3}{*}{I/O} & \multirow{3}{*}{Timer 4 Captura/Comparação/PWM 1}\\
		& 96  & PB1 (3) &     &                                 \\ 
		& 128 & PD7 (3) &     &                                 \\ \hline
		\multirow{2}{*}{T5CCP0}    & 72  & PM6 (3) & \multirow{2}{*}{I/O} & \multirow{2}{*}{Timer 5 Captura/Comparação/PWM 0}\\
		& 91  & PB2 (3) &     &                                 \\ \hline
		\multirow{2}{*}{T5CCP1}    & 71  & PM7 (3) & \multirow{2}{*}{I/O} & \multirow{2}{*}{Timer 5 Captura/Comparação/PWM 1}\\
		& 92  & PB3 (3) &     &                                 \\ \hline
		\caption{Canais Temporizador de Propósito Geral - Tiva TM4C1294NCPDT \cite{DATASHEET_TIVA} }
		\label{tab:CanaisTimer}
	\end{longtable}
\end{center}

\section{Na TivaWare}

As principais funções responsáveis pela configuração e utilização do temporizador de propósito geral, presentes na TivaWare, estão listadas a seguir.

\begin{lstlisting}[style=funcao]
	void TimerConfigure(uint32_t ui32Base,
						uint32_t ui32Config)
\end{lstlisting}

Configura o temporizador especificado.

\begin{description}
	\item [\ttbu{ui32Base}]\hfill \\
	Base do temporizador a ser configurado. Normalmente \textbf{TIMER\emph{k}\_BASE}, onde \textbf{\emph{k}} é o valor identificador do temporizador.
	
	\item [\ttbu{ui32Config}]\hfill \\
	Pacote de parâmetros em formato de OU binário que configura o modo de operação do temporizador especificado. Cada parâmetro é representado no formato \textbf{TIMER\_CFG\_\emph{k}}, para o caso de ser utilizado a base como um único temporizador, ou, \textbf{TIMER\_CFG\_A\_\emph{k}} ou \textbf{TIMER\_CFG\_B\_\emph{k}} para o caso de ser utilizado como dois temporizadores separados. Onde \textbf{\emph{k}} pode assumir os valores:
	\begin{itemize}
		\item \textbf{ONE\_SHOT} temporizador de um disparo.
		\item \textbf{ONE\_SHOT\_UP} temporizador de um disparo, com contador incremental.
		\item \textbf{PERIODIC} temporizador periódico.
		\item \textbf{PERIODIC\_UP} temporizador periódico, com contador incremental.
		\item \textbf{RTC} temporizador com \emph{clock} de tempo real.
		\item \textbf{SPLIT\_PAIR} dois temporizador de meia largura.
		\item \textbf{CAP\_COUNT} temporizador com contador por captura de borda.
		\item \textbf{CAP\_COUNT\_UP} temporizador com contador por captura de borda, com contador incremental.
		\item \textbf{CAP\_TIME} temporizador com tempo por captura de borda.
		\item \textbf{CAP\_TIME\_UP} temporizador com tempo por captura de borda, com contador incremental.
		\item \textbf{CAP\_PWM} temporizador com saída de PWM.
	\end{itemize}
\end{description}

\begin{lstlisting}[style=funcao]
	void TimerEnable(uint32_t ui32Base,
					 uint32_t ui32Timer)
\end{lstlisting}

Habilita a contagem no temporizador especificado.

\begin{description}
	\item [\ttbu{ui32Base}]\hfill \\
	Base do temporizador a ser habilitado. Normalmente \textbf{TIMER\emph{k}\_BASE}, onde \textbf{\emph{k}} é o valor identificador do temporizador.
	
	\item [\ttbu{ui32Timer}]\hfill \\
	Valor que especifica quais temporizadores serão habilitados. Podendo assumir um de 3 valores:
	\begin{itemize}
		\item \textbf{TIMER\_A} para habilitar somente o temporizador A.
		\item \textbf{TIMER\_B} para habilitar somente o temporizador B.
		\item \textbf{TIMER\_BOTH} para habilitar ambos os temporizadores.
	\end{itemize}
\end{description}

\begin{lstlisting}[style=funcao]
	void TimerDisable(uint32_t ui32Base,
					  uint32_t ui32Timer)
\end{lstlisting}

Desabilita a contagem no temporizador especificado.

\begin{description}
	\item [\ttbu{ui32Base}]\hfill \\
	Base do temporizador a ser desabilitado. Normalmente \textbf{TIMER\emph{k}\_BASE}, onde \textbf{\emph{k}} é o valor identificador do temporizador.
	
	\item [\ttbu{ui32Timer}]\hfill \\
	Valor que especifica quais temporizadores serão desabilitados. Podendo assumir um de 3 valores:
	\begin{itemize}
		\item \textbf{TIMER\_A} para habilitar somente o temporizador A.
		\item \textbf{TIMER\_B} para habilitar somente o temporizador B.
		\item \textbf{TIMER\_BOTH} para habilitar ambos os temporizadores.
	\end{itemize}
\end{description}

\begin{lstlisting}[style=funcao]
	void TimerControlLevel(uint32_t ui32Base,
						   uint32_t ui32Timer,
						   bool bInvert)
\end{lstlisting}

Controla o nível de saída do temporizador.

\begin{description}
	\item [\ttbu{ui32Base}]\hfill \\
	Base do temporizador a ser configurado. Normalmente \textbf{TIMER\emph{k}\_BASE}, onde \textbf{\emph{k}} é o valor identificador do temporizador.
	
	\item [\ttbu{ui32Timer}]\hfill \\
	Valor que especifica quais temporizadores serão configurados. Podendo assumir um de 3 valores:
	\begin{itemize}
		\item \textbf{TIMER\_A} para habilitar somente o temporizador A.
		\item \textbf{TIMER\_B} para habilitar somente o temporizador B.
		\item \textbf{TIMER\_BOTH} para habilitar ambos os temporizadores.
	\end{itemize}
	
	\item [\ttbu{bInvert}]\hfill \\
	Se \emph{true}, o sinal de saída é dado por nível baixo, caso contrário, por nível baixo.
\end{description}

\begin{lstlisting}[style=funcao]
	void TimerControlTrigger(uint32_t ui32Base,
							 uint32_t ui32Timer,
							 bool bEnable)
\end{lstlisting}

Habilita ou desabilita o gatilho de saída do temporizador.

\begin{description}
	\item [\ttbu{ui32Base}]\hfill \\
	Base do temporizador a ser configurado. Normalmente \textbf{TIMER\emph{k}\_BASE}, onde \textbf{\emph{k}} é o valor identificador do temporizador.
	
	\item [\ttbu{ui32Timer}]\hfill \\
	Valor que especifica quais temporizadores serão configurados. Podendo assumir um de 3 valores:
	\begin{itemize}
		\item \textbf{TIMER\_A} para habilitar somente o temporizador A.
		\item \textbf{TIMER\_B} para habilitar somente o temporizador B.
		\item \textbf{TIMER\_BOTH} para habilitar ambos os temporizadores.
	\end{itemize}
	
	\item [\ttbu{bEnable}]\hfill \\
	Se \emph{true}, o sinal de gatilho do sinal de saída é habilitado, caso contrário, não.
\end{description}

\begin{lstlisting}[style=funcao]
	void TimerControlEvent(uint32_t ui32Base,
						   uint32_t ui32Timer,
						   uint32_t ui32Event)
\end{lstlisting}

Controla o tipo do evento de saída do temporizador.

\begin{description}
	\item [\ttbu{ui32Base}]\hfill \\
	Base do temporizador a ser configurado. Normalmente \textbf{TIMER\emph{k}\_BASE}, onde \textbf{\emph{k}} é o valor identificador do temporizador.
	
	\item [\ttbu{ui32Timer}]\hfill \\
	Valor que especifica quais temporizadores serão configurados. Podendo assumir um de 3 valores:
	\begin{itemize}
		\item \textbf{TIMER\_A} para habilitar somente o temporizador A.
		\item \textbf{TIMER\_B} para habilitar somente o temporizador B.
		\item \textbf{TIMER\_BOTH} para habilitar ambos os temporizadores.
	\end{itemize}
	
	\item [\ttbu{ui32Event}]\hfill \\
	Valor que especifica o tipo do evento. É definido no formato \textbf{TIMER\_EVENT\_\emph{k}}, onde \textbf{\emph{k}} pode assumir os valores:
	\begin{itemize}
		\item \textbf{POS\_EDGE} para evento disparado na borda positiva.
		\item \textbf{NEG\_EDGE} para evento disparado na borda negativa.
		\item \textbf{BOTH\_EDGES} para evento disparado em ambas as bordas.
	\end{itemize}
\end{description}

\begin{lstlisting}[style=funcao]
	void TimerLoadSet(uint32_t ui32Base,
					  uint32_t ui32Timer,
					  uint32_t ui32Value)
\end{lstlisting}

Configura o valor máximo do temporizador.

\begin{description}
	\item [\ttbu{ui32Base}]\hfill \\
	Base do temporizador a ser configurado. Normalmente \textbf{TIMER\emph{k}\_BASE}, onde \textbf{\emph{k}} é o valor identificador do temporizador.
	
	\item [\ttbu{ui32Timer}]\hfill \\
	Valor que especifica quais temporizadores serão configurados. Podendo assumir um de 3 valores:
	\begin{itemize}
		\item \textbf{TIMER\_A} para habilitar somente o temporizador A.
		\item \textbf{TIMER\_B} para habilitar somente o temporizador B.
		\item \textbf{TIMER\_BOTH} para habilitar ambos os temporizadores.
	\end{itemize}
	
	\item [\ttbu{ui32Value}]\hfill \\
	Valor a ser carregado no temporizador.
\end{description}

\begin{lstlisting}[style=funcao]
	uint32_t TimerLoadGet(uint32_t ui32Base,
						  uint32_t ui32Timer)
\end{lstlisting}

Lê o valor máximo do temporizador.

\begin{description}
	\item [\ttbu{ui32Base}]\hfill \\
	Base do temporizador a ser lido. Normalmente \textbf{TIMER\emph{k}\_BASE}, onde \textbf{\emph{k}} é o valor identificador do temporizador.
	
	\item [\ttbu{ui32Timer}]\hfill \\
	Valor que especifica qual temporizador será lido. Podendo assumir um de 3 valores:
	\begin{itemize}
		\item \textbf{TIMER\_A} para habilitar somente o temporizador A.
		\item \textbf{TIMER\_B} para habilitar somente o temporizador B.
		\item \textbf{TIMER\_BOTH} para habilitar ambos os temporizadores.
	\end{itemize}
\end{description}

\begin{lstlisting}[style=funcao]
	void TimerMatchSet(uint32_t ui32Base,
					   uint32_t ui32Timer,
					   uint32_t ui32Value)
\end{lstlisting}

Configura o valor de comparação carregado no temporizador.

\begin{description}
	\item [\ttbu{ui32Base}]\hfill \\
	Base do temporizador a ser lido. Normalmente \textbf{TIMER\emph{k}\_BASE}, onde \textbf{\emph{k}} é o valor identificador do temporizador.
	
	\item [\ttbu{ui32Timer}]\hfill \\
	Valor que especifica qual temporizador será lido. Podendo assumir um de 3 valores:
	\begin{itemize}
		\item \textbf{TIMER\_A} para habilitar somente o temporizador A.
		\item \textbf{TIMER\_B} para habilitar somente o temporizador B.
		\item \textbf{TIMER\_BOTH} para habilitar ambos os temporizadores.
	\end{itemize}
	
	\item [\ttbu{ui32Valor}]\hfill \\
	Valor no qual o contador irá parar ou, atualizar sua saída.
\end{description}

\begin{lstlisting}[style=funcao]
	uint32_t TimerValueGet(uint32_t ui32Base,
						   uint32_t ui32Timer)
\end{lstlisting}

Lê o valor atual do contador do temporizador.

\begin{description}
	\item [\ttbu{ui32Base}]\hfill \\
	Base do temporizador a ser lido. Normalmente \textbf{TIMER\emph{k}\_BASE}, onde \textbf{\emph{k}} é o valor identificador do temporizador.
	
	\item [\ttbu{ui32Timer}]\hfill \\
	Valor que especifica qual temporizador será lido. Podendo assumir um de 3 valores:
	\begin{itemize}
		\item \textbf{TIMER\_A} para habilitar somente o temporizador A.
		\item \textbf{TIMER\_B} para habilitar somente o temporizador B.
		\item \textbf{TIMER\_BOTH} para habilitar ambos os temporizadores.
	\end{itemize}
\end{description}

\begin{lstlisting}[style=funcao]
	void TimerIntRegister(uint32_t ui32Base,
						  uint32_t ui32Timer,
						  void (*pfnHandler)(void))
\end{lstlisting}

Configura a rotina de tratamento de interrupção do temporizador.

\begin{description}
	\item [\ttbu{ui32Base}]\hfill \\
	Base do temporizador a ser lido. Normalmente \textbf{TIMER\emph{k}\_BASE}, onde \textbf{\emph{k}} é o valor identificador do temporizador.
	
	\item [\ttbu{ui32Timer}]\hfill \\
	Valor que especifica qual temporizador será lido. Podendo assumir um de 3 valores:
	\begin{itemize}
		\item \textbf{TIMER\_A} para habilitar somente o temporizador A.
		\item \textbf{TIMER\_B} para habilitar somente o temporizador B.
		\item \textbf{TIMER\_BOTH} para habilitar ambos os temporizadores.
	\end{itemize}
	
	\item [\ttbu{pfnIntHandler}]\hfill \\
	Ponteiro da função de tratamento. Esta não deve receber nada como parâmetro e nem retornar nada.
\end{description}

\begin{lstlisting}[style=funcao]
	void TimerIntEnable(uint32_t ui32Base,
						uint32_t ui32IntFlags)
\end{lstlisting}

Habilita as interrupções do temporizador.

\begin{description}
	\item [\ttbu{ui32Base}]\hfill \\
	Base do temporizador a ser lido. Normalmente \textbf{TIMER\emph{k}\_BASE}, onde \textbf{\emph{k}} é o valor identificador do temporizador.
	
	\item [\ttbu{ui32IntFlags}]\hfill \\
	Pacote de parâmetros em formato de OU binário que especifica os eventos que podem causar interrupção. Cada parâmetro é definido no formato \textbf{TIMER\_\emph{k}},onde \textbf{\emph{k}} pode assumir os valores:
	\begin{itemize}
		\item \textbf{TIMB\_DMA} quando o uDMA do temporizador B estiver completo.
		\item \textbf{TIMA\_DMA} quando o uDMA do temporizador A estiver completo.
		\item \textbf{CAPA\_EVENT} interrupção por evento de captura do temporizador A.
		\item \textbf{CAPA\_MATCH} interrupção por comparação de captura do temporizador A.
		\item \textbf{TIMA\_TIMEOUT} interrupção por estouro de tempo do temporizador A.
		\item \textbf{CAPB\_EVENT} interrupção por evento de captura do temporizador B.
		\item \textbf{CAPB\_MATCH} interrupção por comparação de captura do temporizador B.
		\item \textbf{TIMB\_TIMEOUT} interrupção por estouro de tempo do temporizador B.
		\item \textbf{RTC\_MATCH} interrupção pela máscara do RTC.
	\end{itemize}
\end{description}

\begin{lstlisting}[style=funcao]
	void TimerIntDisable(uint32_t ui32Base,
						 uint32_t ui32IntFlags)
\end{lstlisting}

Habilita as interrupções do temporizador.

\begin{description}
	\item [\ttbu{ui32Base}]\hfill \\
	Base do temporizador a ser lido. Normalmente \textbf{TIMER\emph{k}\_BASE}, onde \textbf{\emph{k}} é o valor identificador do temporizador.
	
	\item [\ttbu{ui32IntFlags}]\hfill \\
	Pacote de parâmetros em formato de OU binário que especifica os eventos que podem causar interrupção. Cada parâmetro é definido no formato \textbf{TIMER\_\emph{k}},onde \textbf{\emph{k}} pode assumir os valores:
	\begin{itemize}
		\item \textbf{TIMB\_DMA} quando o uDMA do temporizador B estiver completo.
		\item \textbf{TIMA\_DMA} quando o uDMA do temporizador A estiver completo.
		\item \textbf{CAPA\_EVENT} interrupção por evento de captura do temporizador A.
		\item \textbf{CAPA\_MATCH} interrupção por comparação de captura do temporizador A.
		\item \textbf{TIMA\_TIMEOUT} interrupção por estouro de tempo do temporizador A.
		\item \textbf{CAPB\_EVENT} interrupção por evento de captura do temporizador B.
		\item \textbf{CAPB\_MATCH} interrupção por comparação de captura do temporizador B.
		\item \textbf{TIMB\_TIMEOUT} interrupção por estouro de tempo do temporizador B.
		\item \textbf{RTC\_MATCH} interrupção pela máscara do RTC.
	\end{itemize}
\end{description}

\begin{lstlisting}[style=funcao]
	void TimerIntClear(uint32_t ui32Base,
					   uint32_t ui32IntFlags)
\end{lstlisting}

Limpa as \emph{flags} de interrupção do temporizador.

\begin{description}
	\item [\ttbu{ui32Base}]\hfill \\
	Base do temporizador a ser lido. Normalmente \textbf{TIMER\emph{k}\_BASE}, onde \textbf{\emph{k}} é o valor identificador do temporizador.
	
	\item [\ttbu{ui32IntFlags}]\hfill \\
	Pacote de parâmetros em formato de OU binário que especifica os eventos que podem causar interrupção. Cada parâmetro é definido no formato \textbf{TIMER\_\emph{k}},onde \textbf{\emph{k}} pode assumir os valores:
	\begin{itemize}
		\item \textbf{TIMB\_DMA} quando o uDMA do temporizador B estiver completo.
		\item \textbf{TIMA\_DMA} quando o uDMA do temporizador A estiver completo.
		\item \textbf{CAPA\_EVENT} interrupção por evento de captura do temporizador A.
		\item \textbf{CAPA\_MATCH} interrupção por comparação de captura do temporizador A.
		\item \textbf{TIMA\_TIMEOUT} interrupção por estouro de tempo do temporizador A.
		\item \textbf{CAPB\_EVENT} interrupção por evento de captura do temporizador B.
		\item \textbf{CAPB\_MATCH} interrupção por comparação de captura do temporizador B.
		\item \textbf{TIMB\_TIMEOUT} interrupção por estouro de tempo do temporizador B.
		\item \textbf{RTC\_MATCH} interrupção pela máscara do RTC.
	\end{itemize}
\end{description}

\section{Exemplo}

Um exemplo simples de configuração do temporizador é mostrado a seguir. Na Seção \ref{sec:exTimer} é apresentado um exemplo mais aprofundado sobre a configuração e utilização do temporizador.

\begin{lstlisting}[style=citacao]
// Configura o Temporizador A como um temporizador de disparo unico,
// e o temporizador B como um contador por captura de borda
TimerConfigure(TIMER0_BASE, (TIMER_CFG_SPLIT_PAIR | TIMER_CFG_A_ONE_SHOT |
	TIMER_CFG_B_CAP_COUNT));

// Configura o contador do temporizador A como de diparo unico
TimerLoadSet(TIMER0_BASE, TIMER_A, 3000);

// Configura o contador do temporizador B para contar em ambas as bordas
TimerControlEvent(TIMER0_BASE, TIMER_B, TIMER_EVENT_BOTH_EDGES);

// Habilita os temporizadores
TimerEnable(TIMER0_BASE, TIMER_BOTH);
\end{lstlisting}


