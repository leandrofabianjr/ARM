%------------------ BIBLIOTECAS ------------------ 
\usepackage[portuguese]{babel}
\usepackage[utf8]{inputenc}
\usepackage[T1]{fontenc}
\usepackage{times}
\usepackage{url}
\usepackage{amsmath,amsfonts,amssymb}
\usepackage{graphicx}
%\usepackage[pdftex]{hyperref}
\usepackage{blindtext}
\usepackage{indentfirst}
\usepackage{fixltx2e}	
\usepackage{multicol}
\usepackage{wrapfig}
\usepackage{pxfonts}
\usepackage{pgfplots}
%\usepackage{hyperref}
\usepackage{tikz}
\usepackage{circuitikz}
\usepackage[colorinlistoftodos]{todonotes}	
\usepackage{cooltooltips}
\hyphenpenalty=10000
\usepackage{amsmath}
\usepackage{sobraep_01,epsfig,subfigure,cite}
\usepackage{ccaption}
\usepackage{fancyhdr}
\usepackage{times}
\usepackage{float}
\usepackage{pdfpages}
\usepackage[numbered]{mcode}
\usepackage{scrextend}

%------------------ DEFINIÇÕES ------------------------------
\hyphenchar\font=-1
\newtheorem{theorem}{Theorem}
\newtheorem{acknowledgement}[theorem]{Acknowledgement}
\newtheorem{algorithm}[theorem]{Algorithm}
\newtheorem{axiom}[theorem]{Axiom}
\newtheorem{case}[theorem]{Case}
\newtheorem{claim}[theorem]{Claim}
\newtheorem{conclusion}[theorem]{Conclusion}
\newtheorem{condition}[theorem]{Condition}
\newtheorem{conjecture}[theorem]{Conjecture}
\newtheorem{corollary}[theorem]{Corollary}
\newtheorem{criterion}[theorem]{Criterion}
\newtheorem{definition}[theorem]{Definition}
\newtheorem{example}[theorem]{Example}
\newtheorem{exercise}[theorem]{Exercise}
\newtheorem{lemma}[theorem]{Lemma}
\newtheorem{notation}[theorem]{Notation}
\newtheorem{problem}[theorem]{Problem}
\newtheorem{proposition}[theorem]{Proposition}
\newtheorem{remark}[theorem]{Remark}
\newtheorem{solution}[theorem]{Solution}
\newtheorem{summary}[theorem]{Summary}
\newenvironment{proof}[1][Proof]{\textbf{#1.} }{\ \rule{0.5em}{0.5em}}
\setcounter{page}{1}
% Configurando layout para mostrar codigos C++
\usepackage{listings}
% Definindo novas cores
\usepackage{color}
\definecolor{mygreen}{rgb}{0,0.6,0}
\definecolor{mygray}{rgb}{0.5,0.5,0.5}
\definecolor{mymauve}{rgb}{0.58,0,0.82}
\lstset{ 
  backgroundcolor=\color{green!10},  
  basicstyle=\ttfamily\small,            
  breakatwhitespace=false,         
  breaklines=true,                 
  captionpos=b,                    
  commentstyle=\color{mygreen},  
  deletekeywords={...},            
  escapeinside={\%*}{*)},  
  extendedchars=true,    
  %frame=single,	                  
  keepspaces=true,                 
  keywordstyle=\color{blue},       
  language=Octave,,                 
  otherkeywords={*,...},          
  numbers=left,                   
  numbersep=5pt,                   
  numberstyle=\ttfamily\small\color{mygray},
  rulecolor=\color{black},         
  showspaces=false,                
  showstringspaces=false,          
  showtabs=false,                  
  stepnumber=2,                    
  stringstyle=\color{mymauve},  
  tabsize=2,	                   
  title=\lstname,
}
%renomeia o titulo dos tipos figure
\renewcommand{\figurename}{Figura}
%renomeia o titulo dos tipos Table
\renewcommand{\tablename}{Tabela} 
\renewcommand{\lstlistingname}{Código} 
\graphicspath{{figuras/}{fig_site/}}

\title{Implementação de metodologias de controle digital em um conversor \textit{Buck}}