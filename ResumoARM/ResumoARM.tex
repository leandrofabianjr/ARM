\documentclass[10pt,a4paper]{article}


%------------------ BIBLIOTECA E DEFINIÇÕES  ---------------------
%------------------ BIBLIOTECAS ------------------ 
\usepackage[portuguese]{babel}
\usepackage[utf8]{inputenc}
\usepackage[T1]{fontenc}
\usepackage{times}
\usepackage{url}
\usepackage{amsmath,amsfonts,amssymb}
\usepackage{graphicx}
%\usepackage[pdftex]{hyperref}
\usepackage{blindtext}
\usepackage{indentfirst}
\usepackage{fixltx2e}	
\usepackage{multicol}
\usepackage{wrapfig}
\usepackage{pxfonts}
\usepackage{pgfplots}
%\usepackage{hyperref}
\usepackage{tikz}
\usepackage{circuitikz}
\usepackage[colorinlistoftodos]{todonotes}	
\usepackage{cooltooltips}
\hyphenpenalty=10000
\usepackage{amsmath}
\usepackage{sobraep_01,epsfig,subfigure,cite}
\usepackage{ccaption}
\usepackage{fancyhdr}
\usepackage{times}
\usepackage{float}
\usepackage{pdfpages}
\usepackage[numbered]{mcode}
\usepackage{scrextend}

%------------------ DEFINIÇÕES ------------------------------
\hyphenchar\font=-1
\newtheorem{theorem}{Theorem}
\newtheorem{acknowledgement}[theorem]{Acknowledgement}
\newtheorem{algorithm}[theorem]{Algorithm}
\newtheorem{axiom}[theorem]{Axiom}
\newtheorem{case}[theorem]{Case}
\newtheorem{claim}[theorem]{Claim}
\newtheorem{conclusion}[theorem]{Conclusion}
\newtheorem{condition}[theorem]{Condition}
\newtheorem{conjecture}[theorem]{Conjecture}
\newtheorem{corollary}[theorem]{Corollary}
\newtheorem{criterion}[theorem]{Criterion}
\newtheorem{definition}[theorem]{Definition}
\newtheorem{example}[theorem]{Example}
\newtheorem{exercise}[theorem]{Exercise}
\newtheorem{lemma}[theorem]{Lemma}
\newtheorem{notation}[theorem]{Notation}
\newtheorem{problem}[theorem]{Problem}
\newtheorem{proposition}[theorem]{Proposition}
\newtheorem{remark}[theorem]{Remark}
\newtheorem{solution}[theorem]{Solution}
\newtheorem{summary}[theorem]{Summary}
\newenvironment{proof}[1][Proof]{\textbf{#1.} }{\ \rule{0.5em}{0.5em}}
\setcounter{page}{1}
% Configurando layout para mostrar codigos C++
\usepackage{listings}
% Definindo novas cores
\usepackage{color}
\definecolor{mygreen}{rgb}{0,0.6,0}
\definecolor{mygray}{rgb}{0.5,0.5,0.5}
\definecolor{mymauve}{rgb}{0.58,0,0.82}
\lstset{ 
  backgroundcolor=\color{green!10},  
  basicstyle=\ttfamily\small,            
  breakatwhitespace=false,         
  breaklines=true,                 
  captionpos=b,                    
  commentstyle=\color{mygreen},  
  deletekeywords={...},            
  escapeinside={\%*}{*)},  
  extendedchars=true,    
  %frame=single,	                  
  keepspaces=true,                 
  keywordstyle=\color{blue},       
  language=Octave,,                 
  otherkeywords={*,...},          
  numbers=left,                   
  numbersep=5pt,                   
  numberstyle=\ttfamily\small\color{mygray},
  rulecolor=\color{black},         
  showspaces=false,                
  showstringspaces=false,          
  showtabs=false,                  
  stepnumber=2,                    
  stringstyle=\color{mymauve},  
  tabsize=2,	                   
  title=\lstname,
}
%renomeia o titulo dos tipos figure
\renewcommand{\figurename}{Figura}
%renomeia o titulo dos tipos Table
\renewcommand{\tablename}{Tabela} 
\renewcommand{\lstlistingname}{Código} 
\graphicspath{{figuras/}{fig_site/}}

\title{Implementação de metodologias de controle digital em um conversor \textit{Buck}}

%------------------ INÍCIO DO DOCUMENTO ---------------------
\begin{document}

%------------------ TÍTULO ----------------------------------
\title{Desenvolvimento em Microcontroladores Baseados em Processadores ARM Cortex-M4 com TivaWare}

%------------------ AUTORES ---------------------------------
\author{Callebe Barbosa,  Leandro Fabian Junior
\vspace{0.2cm}
\\ Orientador: Gustavo Weber Denardin
\vspace{0.2cm}
\\\normalsize Universidade Tecnológica Federal do Paraná - Campus Pato Branco}

\maketitle

%------------------ RESUMO ---------------- 
\vspace{10pt} {\textbf{Resumo -- Este material apresenta ao leitor conceitos básicos sobre a arquitetura ARM e suas características. Em seguida, os principais periféricos e recursos são introduzidos, explicando seus modos de configuração e utilização. Para as implementações é usado como plataforma de trabalho o Tiva\textsuperscript{TM} - TM4C129NCPDT. Neste material se encontram explicados e exemplificados os recursos básicos de um sistema microcontrolado. São abordados o sistema de clock, portas de entrada e saída de propósito geral, comunicação SPI e UART, conversor analógico/digital, modulação PWM e temporizador de propósito geral. Para facilitar o entendimento dos tópicos, cada um dos recursos apresentados precede um exemplo de configuração implementado em liguagem C. Ao fim são elaborados projetos práticos envolvendo tais recursos.}}


%------------------ abstract ---------------- 
\vspace{10pt} {\textbf{Abstract -- }}

\section{Pré-requisitos}
Para que se possa usufruir de todos os conceitos e práticas presentes neste material, se faz necessário um conhecimento prévio, por parte do leitor, sobre as áreas de microcontrolados e arquitetura de computadores. Bem como uma base em programação, especialmente na linguagem C.

\section{Orientações Pedagógicas}

Os capítulos 2 e 3 se destinam a introduzir os conhecimentos sobre a arquitetura, plataforma e bibliotecas a serem utilizadas nos demais capítulos, servindo como prelúdio. O capítulo 2  destina-se a introduzir o processador ARM, explicando inicialmente suas principais características, modos de operação e registradores internos. Já o capitulo 3 apresenta um resumo sobre os principais periféricos presentes na plataforma de trabalho, o Tiva\textsuperscript{TM} TM4C1294NCPDT.

Em seguida têm-se os capítulos 4 e 5, que apresentam a IDE e as bibliotecas de \emph{software} necessárias e o procedimento para utilizá-las. O capítulo 4 guia o leitor na tarefa de  iniciar um novo projeto na IDE \emph{Code Composer}. Para que se possa utilizar tais bibliotecas de \emph{software}, o capítulo 5 se encarrega de mostrar o passo a passo de como incluí-las ao novo projeto.

Já os capítulos 6 ao 12 sempre seguem o roteiro de apresentar uma introdução teórica sobre o periférico, em seguida demonstrar como este funciona no Tiva\textsuperscript{TM} TM4C1294NCPDT e como é feita sua implementação através das bibliotecas de \emph{software}. Ao fim de cada capítulo, há um exemplo prático de configuração de cada periférico. 

Ao final do material, no capítulo 12, são apresentados exemplos de aplicações utilizando os  periféricos demonstrados ao longo dos capítulos anteriores.  

\vspace{0.5cm}

%\label{Callebe}\textbf{\underline{Callebe Soares Barbosa}} Acadêmica do $8^{\circ}$ período do curso de Engenharia Elétrica da Universidade Tecnológica Federal do Paraná. Email: \url{callebe@alunos.utfpr.edu.br}

%\label{Leandro}\textbf{\underline{Leandro Fabian Junior}} Acadêmico do $8^{\circ}$ período do curso de Engenharia da Computação da Universidade Tecnológica Federal do Paraná. Email: \url{leandrofabianjr@gmail.com}

%------------------ FIM DO DOCUMENTO ------------------------
\end{document}